\subsubsection{Iteratie 3 - Analyse + Realisatie fase}
Het onderzoek naar de blockchain is in iteratie 3 afgerond. Er is voldoende informatie om een geïnformeerd besluit te maken op de juiste oplossingsrichting voor de blockchain technologie. \par

Tijdens deze iteratie en komende iteraties zijn Software Architecturele besluiten gemaakt met een techniek die mij is geleerd tijdens de Software Architectuur lessen in het ASD semester. De techniek houdt in dat de requirements die invloed hebben op de architectuur en andere invloeden, zogenaamde decision forces, opgeschreven en meegenomen worden met een architecturaal besluit. Deze decision forces zijn samen met de product owner opgesteld en met ieder oplossingsrichting meegenomen. (SD-2, SD-3)\par

In deze iteratie gaat het vooral om het besluit van Blockchain framework waarop het proof of concept opgebouwd zou worden. Door het onderzoek naar dit onderwerp in vorige iteraties kan hier vrij snel een besluit opgenomen worden. Het type blockchain dat handig is voor het proof of concept is al bekend: consortium. Hierin is er maar een blockchain technologie die voldoende functionaliteit en documentatie biedt om het proof of concept te bouwen.\par

Voor het modelleren van de architectuur is er gekozen voor de Simon Brown’s C4 Framework. Hier is in deze iteratie een start mee gemaakt waarmee ik 4 verschillende diagrammen kon ontwikkelen met variërende niveaus van abstractie.\par

Naast alle analyse is er ook een stuk software ontwikkeld in deze iteratie. Dit is namelijk het smart contract geschreven in Solidity waarmee de core functionaliteiten en business rules in zijn opgenomen. Zie bijlage A die in de eindfase alle belangrijke requirements R2, R3, R4, R6, R7, R8, R9, R10, R11, R13 en R14 opneemt en wordt uitgevoerd in de blockchain. Deze zijn vervolgens ge-unittest (SD-4, SD-5, SD-6, SD-8).\par

Tijdens de retrospective van deze iteratie is er besloten om de laatste twee iteraties meer focus te zetten op het bereiken van resultaten van het proof of concept.
