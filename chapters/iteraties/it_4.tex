\subsubsection{Iteratie 4 - Analyse + Realisatie fase}
Tijdens de vierde iteratie is het smart contract verder ontwikkeld. Hierbij zijn functionaliteiten zoals het automatisch accepteren van een claim geïmplementeerd. Het probleem hiermee is alleen dat dit nog niet toegankelijk is voor een eindgebruiker. \par

Daarnaast wordt duidelijk vanuit Allianz dat een nieuwe requirement (R1) gebruikers de mogelijkheid moet geven om in te kunnen loggen op het systeem via een webapplicatie. Daarom is er in deze iteratie besloten om te werken aan een simpele web api met gebruiker niveaus, zodat het smart contract in een later stadium gebruikt kan worden door een Grafische gebruikersinterface. (SD-1, SD-4, SD-6, SD-8)\par

Zodoende is er tijdens deze fase gekeken naar een geschikte back-end framework. Hierbij is vooral rekening gehouden met het feit dat er een Blockchain API beschikbaar moet zijn waarmee functies binnen het smart contract aangeroepen kunnen worden. Een ander belangrijk decision force voor dit besluit, is de bekendheid met het framework en taal, om zo de ontwikkelingskosten lager te houden. Hieruit is het Sails JS framework gekozen in combinatie met Angular doordat het geschreven is in Javascript die mogelijkheid biedt om gebruik te maken van de officiële Ethereum JavaScript API, Web3. \par

In deze iteratie is de software architectuur bijgewerkt met de nieuwe besluiten (SD-3). Een ander besluit wordt gemaakt op het moment dat ik erachter kom dat het itereren door een lijst van objecten, zoals een lijst van claims in een smart contract niet mogelijk is. De oplossing hiervoor is het gebruik van een database die op basis van events de aggregatie van data in transacties toegankelijk maakt. Dit besluit is ook met decision forces besloten en doordat er een link naar de transactie (etherscan.io) blijft binnen de blockchain kan de verificatie en validatie uitgevoerd worden.
