\subsubsection{Iteratie 2 - Onderzoek naar de architectuur van de blockchain en smart contracts}
Tijdens de tweede iteratie is er verder onderzoek uitgevoerd naar de blockchain technologie en is het uitgebreid naar Smart Contracts en de blockchain zijn architectuur. Dit allemaal om duidelijk te krijgen hoe hiermee een gedecentraliseerde blockchain applicatie (Dapps \footnote{https://en.wikipedia.org/wiki/Decentralized\_application}) ontwikkeld zou kunnen worden.\par

Verder is het onderzoek aangevuld door bij het onderzoek naar de verschillende types blockchain te kijken wat handig is voor het proof of concept en welke populaire blockchain projecten een gedecentraliseerde applicatie op gebouwd zou kunnen worden. (SD-7)\par

Tijdens deze iteratie is de analyse naar welke factoren in het eindproduct verwerkt moet worden afgerond en vervolgens goedgekeurd door zowel de klant als de product owner. Dit is gedaan door de scope van het proof of concept met de MoSCoW analyse te definiëren.\par

Hieruit is gekozen dat de Must Haves betrekking hebben op het registeren van Claim aanvraag. Deze ziet op het moment dat iedere verzekeraar hierop zijn stem heeft ingevoerd of de claim geaccepteerd of geweigerd wordt. Verder was een Must Have dat dit allemaal door Blockchain transacties inzichtelijk en controleerbaar moet zijn voor iedere gebruiker van het systeem. Op het moment dat dit niet mogelijk is dan heeft het proof of concept een negatiever resultaat voor Allianz, want het doel was zonder een centraal punt alle transacties gevalideerd kunnen worden.\par

Should haves waren bijvoorbeeld om het werk te verminderen dat claim aanvragen met een bedrag onder 1000 euro en het type diefstal automatisch geaccepteerd zouden worden, mits hier voldoende tijd voor is. Daarnaast zijn functionaliteiten zoals het kunnen inloggen met een gebruikersnaam en wachtwoord opgenomen. De Could Haves, de factoren die een leuke toevoeging zouden zijn wanneer er voldoende tijd over is maar dat helaas niet het geval is, is dat polissen en nieuwe makelaars geregistreerd kunnen worden. (SD-1)\par

Daarnaast is in eerste instantie het plan dat het proof of concept gebruik zou maken van het bestaande E-ABS systeem. Na overleg met de stakeholders is duidelijk geworden in deze iteratie dat dit niet het doel is en dat het proof of concept alleen moet aantonen dat de use cases mogelijk zijn met de Blockchain technologie. Dit is ook gelijk een constraint waardoor er tijdens dit project niet gekeken hoeft te worden naar andere methodes dan de blockchain. Dit zou eventueel in een vervolgactie opgenomen kunnen worden.