\chapter{Conclusie \& Aanbevelingen}
In dit hoofdstuk kan aan de hand van dit gehele project de vraag “In hoeverre ben ik geslaagd in het gewenste projectresultaat en daarmee de doelstelling(en) van het project te behalen” worden beantwoord en de belangrijkste conclusies getrokken worden.

Het claimproces van co-insured verzekeringen duurt te lang. Het doel is om aan te tonen dat met de recent opkomende blockchain technologie bestaande bedrijfsprocessen geautomatiseerd kunnen worden. Het resultaat is de responstijd van een claim te verminderen en dit aan te tonen met een proof of concept.

Na afloop van het project kunnen alle hoofdfunctionaliteiten van het claim systeem uitgevoerd worden. Hieruit kan geconcludeerd worden dat de doelstelling van het project is behaald. Het proof of concept is in week 23 gepresenteerd aan de opdrachtgever, Allianz waarop positieve feedback is ontvangen.

Gedurende het gehele project is naar voren gekomen dat de technieken die gebruikt worden met de Blockchain technologie daadwerkelijk functioneel zijn, maar niet altijd even goed meewerken. Het kost veel tijd en uitzoekwerk en het grootste knelpunt zit voornamelijk in de documentatie en support om een gedecentraliseerde blockchain applicatie te realiseren.

Gesteld kan worden dat meer tijd nodig is om een goed werkend product neer te kunnen zetten. De API van de Blockchain verandert ook nog steeds en de applicatie die 2 weken geleden was ontwikkeld, is nu al niet meer compatible met de laatste versie van Ethereum. Hieruit kan ook geconcludeerd worden dat de techniek nog te nieuw is voor functioneel gebruik.

Verder is de conclusie die betrekking heeft tot mijn proces dat ik positieve feedback heb ontvangen van mijn bedrijfsbegeleider Daniël. Ik heb van een lastig project met veel onduidelijkheden en onstabiele techniek een goed resultaat geboekt en mijn proces juist en inzichtelijk uitgevoerd. Hieruit is te concluderen dat ik geslaagd ben in het behalen van het gewenste projectresultaat en de competenties vanuit school. 

\subsection{Aanbeveling}
Aan de hand van de bovenstaande conclusies kan er een aanbeveling gedaan worden. Gezien het feit dat na bijna zes maanden nog geen stabiel prototype gepresenteerd kan worden aan de opdrachtgever, is het aan te raden om dit door te ontwikkelen. Waar tijdens dit project niet naar gekeken hoeft te worden, is het gebruik van andere methodes dan de blockchain. Dit zou eventueel in een vervolgactie opgenomen kunnen worden.