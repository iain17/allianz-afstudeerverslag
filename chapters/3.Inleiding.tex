\chapter{Inleiding}
Voor u ligt het afstudeerverslag van Iain Munro, dat is geschreven naar aanleiding van het afstudeerproject dat loopt van 5 februari 2018 tot en met 29 juni 2018 (2017/2018 periode 3 \& 4). De Afstudeeropdracht is uitgevoerd in opdracht van Allianz Group via het stagebedrijf HeadFWD. Deze opdracht bestond uit het ontwikkelen van een ’proof of concept’ voor de opdrachtgever, waarbij vooraf analyse en onderzoek werd uitgevoerd.
\par 
Het doel van dit verslag is om inzicht te geven over de inhoud, aanpak, verloop en resultaten van dit project. Het eerste gedeelte van dit verslag heeft betrekking op de context van het project. Hierna volgt de opdrachtomschrijving, waarin kort de opdracht en het uitgevoerde onderzoek worden beschreven. Aangezien dit al eerder uitgebreid is beschreven wordt hier alleen kort samengevat met een verwijzing naar het Plan van Aanpak.
\par
Vervolgens worden de gebruikte methoden en technieken besproken. Hierbij wordt voornamelijk gekeken naar de reden waarom deze technieken zijn gebruikt, en in hoeverre deze nuttig waren voor het project. Verder staat er een korte beschrijving van alle (tussen)resultaten van het project.
\par
In de procesbeschrijving worden de belangrijkste activiteiten en besluiten verantwoord. Hierbij wordt verwezen naar de verschillende competenties die de bekwaamheid aantonen van de afstudeerder als beginnend beroepsprofessional. Verder worden de opgeleverde resultaten en producten toegelicht. Hierdoor zou er een duidelijk beeld moeten ontstaan en zou het helder moeten zijn in hoeverre deze voldoen aan de gestelde eisen. Het volledige eindproduct is meegeleverd met dit document en een verwijzing is te vinden in de bijlagen.
\par
Dit verslag wordt afgerond met de conclusies en aanbevelingen. Hierin wordt teruggekoppeld naar de opdracht- en doelstelling van het project. De vraag “In hoeverre ben ik geslaagd in het gewenste projectresultaat en daarmee de doelstelling(en) van het project te behalen” wordt hier beantwoord, en de belangrijkste conclusies worden weergegeven. Op de momenten waar het gaat over mijn proces maak ik gebruik van de ik vorm, omdat dit gaat over mijn proces. Naar aanleiding van deze conclusies wordt er een aanbeveling gedaan voor eventuele vervolgacties of aandachtspunten voor het opgeleverde resultaat.
\par
Het einde van dit verslag bestaat uit een bronnenlijst waarbij de opgeleverde producten zoals het Plan van Aanpak, Onderzoeksverslag en Analyse documenten zijn meegeleverd met dit verslag. 
